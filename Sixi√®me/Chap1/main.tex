\documentclass[12pt]{article}
\usepackage{eso-pic,graphicx}
\usepackage{geometry}
\usepackage{amsmath,amsthm,amssymb,amsfonts}
\usepackage{fontspec}


\newtheorem{definition}{Définition}
\newtheorem*{question}{Question}
\newtheorem{theorem}{Théorème}
\newtheorem*{examples}{Exemples}
\newtheorem*{example}{Exemple}

\begin{document}

\part*{Nombres entiers et décimaux}
\section{C'est quoi un nombre ?}

\paragraph{}
D'après Wikipedia :

\begin{definition}
    Un nombre est un concept permettant d’évaluer et de comparer des quantités ou des rapports de grandeurs.
\end{definition}

\paragraph{}
Ne vous embêtez pas à apprendre cette définition, mais retenez surtout qu'un nombre n'est rien de plus qu'un concept abstrait.

\paragraph{}
Par exemple, supposons que je vous pose la question suivante dans un devoir :

\begin{question}
  Combien de nains accompagnent Blanche-Neige ?
\end{question}

et que vous m'écrivez :
$$7$$

ou

$$sept$$

ou

$$3+4$$

ou

$$sieben$$

ou

$$\frac{68607}{2 \pi \sqrt{2} \sum_{n = 0}^{+ \infty}\frac{(1103 + 26390n)n!}{(396^{n}n!)^{4}}}$$

\paragraph{}
Alors vous aurez donné \textbf{à chaque fois} la bonne réponse.

\paragraph{}
Seulement, chaque réponse présuppose certaines connaissances de ma part:
\begin{itemize}
  \item $\frac{68607}{2 \pi \sqrt{2} \sum_{n = 0}^{+ \infty}\frac{(1103 + 26390n)n!}{(396^{n}n!)^{4}}}$ suppose que j'ai fait un peu de maths
  \item $sieben$ suppose que je connais un peu la langue allemande
  \item $3+4$ suppose que je sais faire des additions
  \item $sept$ suppose que je connais un peu la langue française
  \item $7$ suppose que je connais \textbf{l'écriture décimale}.
\end{itemize}

\paragraph{}
Comme vous pouvez le voir, il existe une multitude de façons d'exprimer un même nombre, certaines écritures sont plus pratiques que d'autres selon la situation, l'écriture décimale ( 7 ) n'en est qu'une parmi \textbf{beaucoup}.


\paragraph{}
Par exemple, une bonne réponse à la question «Combien de nains accompagnent Blanche-Neige ?» aurait été «Le nombre de nains qui accompagnent Blanche-Neige». C'est rigolo comme réponse, mais ça ne nous fait pas beaucoup avancer…

\paragraph{}
\emph{Dans la vraie vie}, pour éviter ce genre de désagréments, on s'attend à ce que la réponse soit donnée en écriture décimale.

\section{L'écriture décimale}

\subsection{Définition}
\begin{definition}
Un nombre est écrit en écriture décimale dès qu'on l'exprime sous la forme d'une série de chiffres suivie \textbf{éventuellement} d'une virgule puis d'une série de chiffres.
\end{definition}

\begin{examples}

  $7$ \textbf{est écrit} en écriture décimale

  $7,0$ \textbf{est écrit} en écriture décimale

  $007,0000$ \textbf{est écrit} en écriture décimale

  $1337,42$ \textbf{est écrit} en écriture décimale

  $3+4$ \textbf{n'est pas écrit} en écriture décimale

  $\frac{70}{10}$ \textbf{n'est pas écrit} en écriture décimale
\end{examples}

\paragraph{}
Les nombres que l'on peut écrire en écriture décimale sont appelés \textbf{nombres décimaux} \footnote{il existe des nombres qu'on ne peut pas écrire en écriture décimale, mais vous verrez ça dans très longtemps !}.

\paragraph{}
Prenons un nombre au hasard, par exemple $4832,326$
\begin{itemize}
  \item $4$ est le chiffre des \textbf{milliers}
  \item $8$ est le chiffre des \textbf{centaines}
  \item $3$ est le chiffre des \textbf{dizaines}
  \item $2$ est le chiffre des \textbf{unités}
  \item $3$ est le chiffre des \textbf{dixièmes}
  \item $2$ est le chiffre des \textbf{centièmes}
  \item $6$ est le chiffre des \textbf{millièmes}
\end{itemize}

C'est plus clair en l'écrivant dans un tableau, chiffre par chiffre :

\paragraph{}
\begin{tabular}{|c|c|c|c|c|c|c|c|}
\hline
milliers & centaines & dizaines & unités & & dixièmes & centièmes & millièmes \\
\hline
4 & 8 & 3 & 2 & , & 3 & 2 & 6 \\
\hline
\end{tabular}

\begin{theorem}
  Si on rajoute des $0$ devant l'écriture décimale d'un nombre, on obtient une autre écriture décimale du même nombre.
\end{theorem}

\begin{example}
  $42,1337$ et $00042,1337$ sont des écritures décimales du même nombre.
\end{example}

\begin{theorem}
  Si on rajoute des $0$ à la fin de l'écriture décimale d'un nombre et que cette écriture décimale possède une virgule, on obtient une autre écriture décimale du même nombre.
\end{theorem}

\begin{example}
  $42,1337$ et $42,13370000$ sont des écritures décimales du même nombre.
\end{example}

\begin{theorem}
  Si on rajoute une virgule suivie d'un ou plusieurs $0$ à la fin de l'écriture décimale d'un nombre et que cette écriture ne possède pas de virgule, on obtient un autre écriture décimale du même nombre.
\end{theorem}

\begin{example}
  $7$ et $7,000000$ sont des écritures décimales du même nombre.
\end{example}

\subsection{Addition}


\begin{example}

Supposons qu'on veuille additionner $342,32$ et $125,61$.

\begin{enumerate}
 
\item on alligne les virgules

\begin{tabular}{c c c c c c c}
  & 3 & 4 & 2 & , & 3 & 2 \\
+ & 1 & 2 & 5 & , & 6 & 1 \\
\hline
=\\
\end{tabular}

\item on additionne les chiffres un à un de droite à gauche

\begin{tabular}{c c c c c c c}
  & 3 & 4 & 2 & , & 3 & 2 \\
+ & 1 & 2 & 5 & , & 6 & 1 \\
\hline
= & 4 & 6 & 7 & , & 9 & 3
\end{tabular}
 

\item et donc $342,32 + 125,61 = 467,93$
\end{enumerate}
\end{example}

\paragraph{}
Mmmmmhhhh, là on a eu de la chance. Que fait-on si le résultat d'une des sommes est supérieure ou égale à $10$ ?

\begin{example}
Si on avait voulu additionner $492,63$ et $331,72$, on aurait dû avoir recours aux retenues

\begin{enumerate}

 
\item on alligne les virgules

\begin{tabular}{c c c c c c c}
  & 4 & 9 & 2 & , & 6 & 3 \\
+ & 3 & 3 & 1 & , & 7 & 2 \\
\hline
=\\
\end{tabular}

\item on additionne les chiffres un à un en partant de la droite. Commençons par les centièmes.

\begin{tabular}{c c c c c c c}
  & 4 & 9 & 2 & , & 6 & 3 \\
+ & 3 & 3 & 1 & , & 7 & 2 \\
\hline
= &  &  &  &  &  & 5
\end{tabular}
 
\item Quand on veut calculer les dixièmes, on tombe sur un petit problème : $6 + 7 = 13$. La solution est de noter $3$ et de retenir $1$ :

\begin{tabular}{c c c c c c c}
  &   &   & 1 &   &   &   \\
  & 4 & 9 & 2 & , & 6 & 3 \\
+ & 3 & 3 & 1 & , & 7 & 2 \\
\hline
= &  &  &  &  & 3  & 5
\end{tabular}

\item On continue sans problème pour les unités, et on retombe sur le même problème pour les dizaines ! Rebelotte, on pose une retenue :

\begin{tabular}{c c c c c c c}
  & 1 &  & 1 &   &   &   \\
  & 4 & 9 & 2 & , & 6 & 3 \\
+ & 3 & 3 & 1 & , & 7 & 2 \\
\hline
= & 8 & 2 & 4 & , & 3  & 5
\end{tabular}
 

\item et donc $492,63 + 331,72 = 824,35 $
\end{enumerate}
\end{example}

Il y a un dernier cas de figure qui peut arriver et auquel il faut penser : que se passe-t'il si le
\subsection{Multiplication}

\section{Les fractions}

\end{document}
